% Created 2019-09-01 日 12:47
% Intended LaTeX compiler: pdflatex
\documentclass[11pt]{article}
\usepackage[utf8]{inputenc}
\usepackage[T1]{fontenc}
\usepackage{graphicx}
\usepackage{grffile}
\usepackage{longtable}
\usepackage{wrapfig}
\usepackage{rotating}
\usepackage[normalem]{ulem}
\usepackage{amsmath}
\usepackage{textcomp}
\usepackage{amssymb}
\usepackage{capt-of}
\usepackage{hyperref}
\author{nicolas4d}
\date{\today}
\title{}
\hypersetup{
 pdfauthor={nicolas4d},
 pdftitle={},
 pdfkeywords={},
 pdfsubject={},
 pdfcreator={Emacs 27.0.50 (Org mode 9.1.11)}, 
 pdflang={English}}
\begin{document}

\tableofcontents

\#+TITLE Org Mode
\section{introduction}
\label{sec:org2054775}
\subsection{Summary}
\label{sec:orge830729}
\subsection{Installation}
\label{sec:org53aa450}
\subsection{Activeation}
\label{sec:org4de31b9}
\subsection{Feedback}
\label{sec:org702cfdd}
\subsection{Typesetting conventions used in this manual}
\label{sec:org0836ecf}
keybindings
C-c a org-agenda
C-c c org-capture
\section{Dcoument structure}
\label{sec:orgd12febb}
\subsection{\^{}}
\label{sec:org797529e}
Org is based on Outline mode
\subsection{Outlines}
\label{sec:orge0eea1a}
<TAB>  org-cycle
\subsection{Headlines}
\label{sec:orge5cce70}
stars * one or more
org-footnote-section, which defaults to Footnotes
org-cycle-separator-lines

‘org-special-ctrl-a/e’, ‘org-special-ctrl-k’, ‘org-ctrl-k-protect-subtree’ 
\subsection{Visibility cycling}
\label{sec:org4ce4483}
\subsubsection{Global and local cycling}
\label{sec:org3e6facc}
‘<TAB>’     (‘org-cycle’)
‘S-<TAB>’     (‘org-global-cycle’)
C-u <TAB>
N + S-<TAB>   show level N 
‘C-u C-u <TAB>’     (‘org-set-startup-visibility’)
‘C-u C-u C-u <TAB>’     (‘outline-show-all’)
‘C-c C-r’     (‘org-reveal’) in sparse trees or agenda commads
‘C-c C-k’     (‘outline-show-branches’)
N + ‘C-c <TAB>’     (‘outline-show-children’)
N + ‘C-c C-x b’     (‘org-tree-to-indirect-buffer’)
‘C-c C-x v’     (‘org-copy-visible’)
\subsubsection{Initial visibility}
\label{sec:org221532f}
org-startup-folded

per-file basis:
VISIBILITY (*note Properties and Columns)
‘C-u C-u <TAB>’     (‘org-set-startup-visibility’)

org-agenda-inhibit-startup
\subsubsection{Catching invisible edits}
\label{sec:org838bed7}
org-catch-invisible-edits
\subsection{Motion}
\label{sec:org7535c8e}
‘C-c C-n’     (‘outline-next-visible-heading’)
‘C-c C-p’     (‘outline-previous-visible-heading’)
‘C-c C-f’     (‘org-forward-same-level’)
‘C-c C-b’     (‘org-backward-same-level’)
‘C-c C-u’     (‘outline-up-heading’)
‘C-c C-j’     (‘org-goto’)
\subsection{Structure editing}
\label{sec:orgad8f5a6}
 ‘M-<RET>’     (‘org-insert-heading’)
 C-u C-u M-<RET>
 ‘C-<RET>’     (‘org-insert-heading-respect-content’)
 ‘M-S-<RET>’     (‘org-insert-todo-heading’) org-treat-insert-todo-heading-as-state-change
 ‘C-S-<RET>’     (‘org-insert-todo-heading-respect-content’)
 ‘<TAB>’     (‘org-cycle’)
 ‘M-<left>’     (‘org-do-promote’)
 ‘M-<right>’     (‘org-do-demote’)
 ‘M-S-<left>’     (‘org-promote-subtree’)
 ‘M-S-<right>’     (‘org-demote-subtree’)
 ‘M-h’     (‘org-mark-element’)
‘C-c @’     (‘org-mark-subtree’)
‘C-c C-x C-w’     (‘org-cut-subtree’)
‘C-c C-x M-w’     (‘org-copy-subtree’)
‘C-c C-x C-y’     (‘org-paste-subtree’)
‘C-y’     (‘org-yank’)
‘C-c C-x c’     (‘org-clone-subtree-with-time-shift’) org-clone-subtree-with-time-shift
‘C-c C-w’     (‘org-refile’)
‘C-c \^{}’     (‘org-sort’)   C-u case-sensitve
‘C-x n s’     (‘org-narrow-to-subtree’)
‘C-x n b’     (‘org-narrow-to-block’)
‘C-x n w’     (‘widen’)
‘C-c *’     (‘org-toggle-heading’)
org-M-RET-may-split-line
\subsection{Sparse trees}
\label{sec:orgfa977b5}
‘C-c /’     (‘org-sparse-tree’)
‘C-c / r’  or  ‘C-c / /’     (‘org-occur’)     C-c C-c disppear    with C-U 
‘M-g n’  or  ‘M-g M-n’     (‘next-error’)
‘M-g p’  or  ‘M-g M-p’     (‘previous-error’)

org-agenda-custom-commands :
(setq org-agenda-custom-commands
           '(("f" occur-tree "FIXME")))

ps-print-buffer-with-faces
C-c C-e C-v

org-show-context-detail
org-remove-highlights-with-change
\subsection{Plain lists}
\label{sec:org947b3e0}
\subsubsection{list}
\label{sec:orgaed4ac0}
\begin{itemize}
\item Unordered: - + *
\item Ordered: 1. 1) [@20]
\item[{Description: unordered item}] description term
\end{itemize}

list must have same left indention.
lists ends:
\begin{enumerate}
\item less or equally indented than items at top level.
\item before tow blank lines.
\end{enumerate}

org-list-demote-modify-bullet
org-list-indent-offset
org-list-automatic-rules
\subsubsection{list keybindings}
\label{sec:org34f917a}
\begin{itemize}
\item ‘<TAB>’     (‘org-cycle’)    
\begin{itemize}
\item fold items
\item new item with no text, change level
\item org-cycle-include-plain-lists
\end{itemize}
\item ‘M-<RET>’     (‘org-insert-heading’)
\item ‘M-S-<RET>’ checkbox
\item ‘S-up’
\item ‘S-down’
\item ‘M-up’
\item ‘M-down’
\item ‘M-S-<LEFT>’
\item ‘M-S-<RIGHT>’
\begin{itemize}
\item org-list-automatic-rules
\end{itemize}
\item ‘C-c C-c’ toggle checkbox
\item ‘C-c -’ cycle bullets
\begin{itemize}
\item org-plain-list-ordered-item-terminator
\item at active region all become list items.
\item C-u region become single item.
\item a normal line converted to a list item.
\end{itemize}
\item ‘C-c *’ to headline
\item ‘C-c C-*’ whole plain list to subtree.
\begin{itemize}
\item checkbox become TODO
\end{itemize}
\item ‘S-<LEFT>/<RIGHT>’ cycles bullet 
\begin{itemize}
\item ‘org-support-shift-select’.
\end{itemize}
\item ‘C-c \^{}’ sort the plain list.
\end{itemize}
\subsubsection{FootnotesInfomation}
\label{sec:org3c04905}
\begin{itemize}
\item '*' must indent
\item org-plain-list-ordered-item-terminator
\item org-list-allow-alphabetical
\setcounter{enumi}{19}
\item before checkbox
\item org-M-RET-may-split-line
\item org-list-use-circular-motion
\item org-list-use-circular-motion
\end{itemize}
\subsection{Drawers}
\label{sec:orgcfe4a3e}
‘C-c C-x d’   ‘org-insert-drawer’
'C-u C-c C-x d'    org-insert-property-drawer
'C-M-i'  completion over drawer
‘C-c C-z’ Add a time-stamped note
‘org-export-with-drawers’
‘org-export-with-properties’ 
\subsection{Blocks}
\label{sec:orge7957fa}
‘org-hide-block-startup’ 
or
\subsection{The Orgstruct}
\label{sec:org15fa10e}
minor mode ‘orgstruct-mode’

‘M-x orgstruct-mode <RET>’
or
in message mode :
(add-hook 'message-mode-hook 'turn-on-orgstruct)
(add-hook 'message-mode-hook 'turn-on-orgstruct++)

orgstruct++-mode
orgstruct-heading-prefix-regexp
\subsection{Org syntax}
\label{sec:orgde8eb12}
\url{https://orgmode.org/worg/dev/org-syntax.html}
org-lint  check syntax
\section{Tables}
\label{sec:orgf33bf5a}
\subsection{Built-in table editor}
\label{sec:orgb3ce69d}
\subsubsection{\^{}}
\label{sec:orgc03304f}
<TAB> re-align and next field
<RET> re-align and next row
C-c C-c re-align
\subsubsection{create table}
\label{sec:orgd54618e}
\begin{center}
\begin{tabular}{lll}
Name & Phone & Age\\
\hline
\end{tabular}
\end{center}
type <TAB>

\begin{center}
\begin{tabular}{lll}
Name & Phone & Age\\
\end{tabular}
\end{center}
type C-c <RET>
\subsubsection{keybindings}
\label{sec:org8bf0781}
\begin{itemize}
\item ‘C-c |     (org-table-create-or-convert-from-region)’
\begin{itemize}
\item convert region to table
\item C-u ‘C-u’ forces CSV, ‘C-u C-u’ forces TAB, ‘C-u C-u C-u’ will prompt for a regular expression.
\item no active region then just create
\end{itemize}
\item ‘C-c C-c     (org-table-align)’
\item ‘C-c <SPC>     (org-table-blank-field)’
\item ‘<TAB>     (org-table-next-field)’
\item ‘S-<TAB>     (org-table-previous-field)’
\item ‘<RET>     (org-table-next-row)’
\item ‘M-a     (org-table-beginning-of-field)’
\item ‘M-e     (org-table-end-of-field)’
\item ‘M-<LEFT>     (org-table-move-column-left)’
\item ‘M-<RIGHT>     (org-table-move-column-right)’
\item ‘M-S-<LEFT>     (org-table-delete-column)’
\item ‘M-S-<RIGHT>     (org-table-insert-column)’
\item ‘M-<UP>     (org-table-move-row-up)’
\item ‘M-<DOWN>     (org-table-move-row-down)’
\item ‘M-S-<UP>     (org-table-kill-row)’
\item ‘M-S-<DOWN>     (org-table-insert-row)’
\item ‘C-c -     (org-table-insert-hline)’
\item ‘C-c <RET>     (org-table-hline-and-move)’
\begin{itemize}
\item + C-c
\end{itemize}
\item ‘C-c C-x M-w     (org-table-copy-region)’
\item ‘C-c C-x C-w     (org-table-cut-region)’
\item ‘C-c C-x C-y     (org-table-paste-rectangle)’
\item ‘M-<RET>     (org-table-wrap-region)’
\begin{itemize}
\item + C-u change the number of desired lines.
\end{itemize}
\item ‘C-c +     (org-table-sum)’
\begin{itemize}
\item inserted with C-y
\end{itemize}
\item ‘S-<RET>     (org-table-copy-down)’
\begin{itemize}
\item integer field values will be incremented. prefix 0 disable increment.
\end{itemize}
\item ‘C-c `     (org-table-edit-field)’
\begin{itemize}
\item in a separate window
\item + C-u full visible
\item + C-u C-u separate window follow the cursor.
\end{itemize}
\item ‘M-x org-table-import <RET>’
\begin{itemize}
\item import a file as a table. separator <TAB>
\item + C-u determine the separator.
\end{itemize}
\item ‘C-c |     (org-table-create-or-convert-from-region)’
\item ‘M-x org-table-export <RET>’
\begin{itemize}
\item ‘org-table-export-default-format’.‘TABLE\(_{\text{EXPORT}}\)\(_{\text{FILE}}\)’ and ‘TABLE\(_{\text{EXPORT}}\)\(_{\text{FORMAT}}\)’
\end{itemize}
\end{itemize}
\subsubsection{footnoteInfo}
\label{sec:orgd149ae1}
(1) To insert a vertical bar into a table field, use ‘\(\vert{}\)’ or,
inside a word ‘abc\(\vert{}\)def’.
\subsection{Column width and alignment}
\label{sec:orgb3d6ada}
set width of a column:
<N>
inside field.

C-c ` 

‘org-startup-align-all-tables’ 
per-file basis with:
alignment:
<r><c><l>
\subsection{Column groups}
\label{sec:org3e7ee2e}
\begin{center}
\begin{tabular}{r|rrr|rr|}
N & N\(^{\text{2}}\) & N\(^{\text{3}}\) & N\(^{\text{4}}\) & \texttt{sqrt(n)} & \texttt{sqrt[4](N)}\\
\hline
1 & 1 & 1 & 1 & 1 & 1\\
\hline
\end{tabular}
\end{center}
\subsection{Orgtbl mode}
\label{sec:org03f4afb}
‘M-x orgtbl-mode <RET>’.

(add-hook 'message-mode-hook 'turn-on-orgtbl)
\subsection{The spreadsheet}
\label{sec:org24419f4}
\subsubsection{References}
\label{sec:org9e89aeb}
\begin{enumerate}
\item \^{}
\label{sec:orgaf18a6f}
C-c ?  coordinates of a field are.
C-c \}  toggle the display of a grid.
\item Field references
\label{sec:org91fe00d}
\begin{itemize}
\item @ROW\$COLUMN
\begin{itemize}
\item column
\begin{itemize}
\item \$0 current
\item \$1 \$2 \ldots{}
\item \$+1 \ldots{}
\item \$< first \$> last
\item \$>>> third column from the right
\end{itemize}
\item row
\begin{itemize}
\item @0 current
\item @1 \ldots{}
\item @+1 \ldots{}
\item @< @>
\item @I first hline(horizontal separator)
\item @II
\item @-1
\item @III+2
\end{itemize}
\end{itemize}
\item references
\begin{itemize}
\item with unsigned numbers fixed.
\item with signed numbers relative.
\end{itemize}
\end{itemize}
\item Range references
\label{sec:org22d0ec9}
\ldots{}
return a vector
\item Field coordinates in formulas
\label{sec:org374a196}
@\# org-table-current-dline
\$\# org-table-current-column
\item Named references
\label{sec:org5c6cc54}
\$name
org-table-formula-constants

\$PROP\(_{\text{Xyz}}\)
\begin{itemize}
\item constants.el
\begin{itemize}
\item \$I \#+STARTUP  constSI
\item cgs \#+STARTUP  constcgs
\item constants-unit-system
\end{itemize}
\end{itemize}
\item Remote references
\label{sec:orgd9d132f}
\begin{itemize}
\item remote(NAME-OR-ID,REF)
\begin{itemize}
\item NAME ‘\#+NAME: Name’ line before the table.
\item NAME @ROW\$COLUMN    this is indirection of NAME-OR-ID.
\end{itemize}
\end{itemize}
\end{enumerate}
\subsubsection{Formula syntax for Calc}
\label{sec:org0dac23f}
a/b*c  ==  a/(b*c)
mode string after a semicolon.
‘org-calc-default-modes’.
\subsubsection{Formula syntax for Lisp}
\label{sec:orgb50ecd8}
'()
\subsubsection{Durations and time values}
\label{sec:orgb2b0f73}
\subsubsection{Field and range formulas}
\label{sec:org19fd4e8}
\subsubsection{Column formulas}
\label{sec:org18998f7}
\subsubsection{Lookup functions}
\label{sec:orge06a89b}
\subsubsection{Editing and debugging formulas}
\label{sec:org89143ff}
\subsubsection{Updating the table}
\label{sec:orga6d4922}
\subsubsection{Advanced features}
\label{sec:org3a1b23b}
\subsection{Org-Plot}
\label{sec:org4d8cb07}
\subsubsection{Graphical plots using ‘Gnuplot’}
\label{sec:orgcfcd711}
\subsubsection{ASCII bar plots}
\label{sec:org8a265ed}
C-c " a
M-x orgtbl-ascii-plot <RET>
\section{Hyperlinks}
\label{sec:org5c1fc01}
\subsection{Link format}
\label{sec:orgc50d120}
[[link][description] ] or [[link] ]   erase ' '
\subsection{Internal links}
\label{sec:org081b1aa}
\begin{itemize}
\item ‘[ [\#my-custom-id]]’ link to ‘CUSTOM\(_{\text{ID}}\)’ property ‘my-custom-id’.
\item \ref{org6f0e3cb} \hyperref[org6f0e3cb]{find my target}
\begin{itemize}
\item C-c C-o follow link
\end{itemize}
\end{itemize}
\label{org6f0e3cb}
search for headline

C-c \&  previous position of mark ring.
Radio targets C-c C-c on target
\subsection{External links}
\label{sec:org179efc9}
[[string:someString] ]    no ' '
\subsection{Handling links}
\label{sec:orgb79d198}
\begin{itemize}
\item ‘C-c l     (org-store-link)’
\item ‘C-c C-l     (org-insert-link)’
\item ‘C-u C-c C-l’  link to file
\item ‘C-u C-u C-c C-l’ link to file as absolute path
\item ‘C-c C-l  (with cursor on existing link)’   edit
\item ‘C-c C-o     (org-open-at-point)’
\begin{itemize}
\item + C-u  with emacs
\item + C-u C-u  avoid with emacs
\item org-link-frame-setup
\end{itemize}
\item ‘<RET>’
\begin{itemize}
\item org-return-follows-link
\end{itemize}
\item ‘C-c C-x C-v     (org-toggle-inline-images)’
\begin{itemize}
\item org-startup-with-inline-images
\item ‘\#+STARTUP’ keywords ‘inlineimages’ and ‘noinlineimages’
\end{itemize}
\item ‘C-c \%     (org-mark-ring-push)’
\item ‘C-c \&     (org-mark-ring-goto)’
\item ‘C-c C-x C-n     (org-next-link)’
\item ‘C-c C-x C-p     (org-previous-link)’
\end{itemize}

\subsection{Using links outside Org}
\label{sec:orga450ba4}
(global-set-key "\C-c L" 'org-insert-link-global)
(global-set-key "\C-c o" 'org-open-at-point-global)
\subsection{Link abbreviations}
\label{sec:orgf86a0fd}
[[linkword:tag][description] ]  with no blank
(setq org-link-abbrev-alist
 '()
)
\%s \%h \%(my-function)
\url{http://www.google.com/search?q=OrgMode}

(org-link-set-parameters ``type'' :complete \#'some-function)
\subsection{Search options}
\label{sec:org3d7e429}
:: 

\url{file:///home/d/code/main.c}
\url{file:///home/d/xx.org}
\url{file:///home/d/xx.org}
[[\url{file:///home/d/xx.org}] ]
\url{file:///home/d/xx.org}
\url{}
\subsection{Custom searches}
\label{sec:org965afd2}
‘org-create-file-search-functions’
‘org-execute-file-search-functions’
\section{{\bfseries\sffamily TODO} items}
\label{sec:orgd4ca0b2}
\subsection{{\bfseries\sffamily TODO} basics}
\label{sec:org4102bb4}
\begin{itemize}
\item ‘C-c C-t     (org-todo)’
\begin{itemize}
\item ‘org-use-fast-todo-selection’
\item ‘C-u C-c C-t’ prompt
\end{itemize}
\item ‘S-<RIGHT>  /  S-<LEFT>’ cycling
\begin{itemize}
\item ‘shift-selection-mode’
\item ‘org-treat-S-cursor-todo-selection-as-state-change’
\end{itemize}
\item ‘C-c / t     (org-show-todo-tree)’  view TODO items
\begin{itemize}
\item C-c / T or C-u  specific TODO
\item C-c N   show the tree Nth keyword in org-todo-keywords
\end{itemize}
\item ‘C-c a t     (org-todo-list)’
\item ‘S-M-<RET>     (org-insert-todo-heading)’
\end{itemize}
org-todo-state-tags-triggers
\subsection{{\bfseries\sffamily TODO} extensions}
\label{sec:orgdc9d6f4}
\subsubsection{\^{}}
\label{sec:org298ded8}
org-todo-keywords
\subsubsection{Workflow states}
\label{sec:org853f4a3}
(setq org-todo-keywords
   '((sequence "TODO" "FEEDBACK" "VERIFY" "|" "DONE" "DELEGATED")))
\subsubsection{{\bfseries\sffamily TODO} types}
\label{sec:org05f7f6a}
(setq org-todo-keywords '((type "Fred" "Sara" "Lucy" "|" "DONE")))
\subsubsection{Multiple sets in one file}
\label{sec:orge76d66a}
(setq org-todo-keywords
      '((sequence "TODO" "|" "DONE")
        (sequence "REPORT" "BUG" "KNOWNCAUSE" "|" "FIXED")
        (sequence "|" "CANCELED")))

‘C-u C-u C-c C-t’
‘C-S-<RIGHT>’
‘C-S-<LEFT>’
     These keys jump from one TODO subset to the next.
‘S-<RIGHT>’
‘S-<LEFT>’
     walk through \uline{all} keywords from all.
shift-selection-mode
\subsubsection{Fast access to TODO states}
\label{sec:orgbfe3480}
     (setq org-todo-keywords
           '((sequence "TODO(t)" "|" "DONE(d)")
             (sequence "REPORT(r)" "BUG(b)" "KNOWNCAUSE(k)" "|" "FIXED(f)")
             (sequence "|" "CANCELED(c)")))
org-fast-tag-selection-include-todo
\subsubsection{Per-file keywords}
\label{sec:orgcdba399}
or
\#+TODO == \#SEQ\(_{\text{TODO}}\)

M-<TAB> completion

C-c C-c  make the changes known to org mode.
\subsubsection{Faces for TODO keywords}
\label{sec:org32296c3}
org-todo 
org-done

(setq org-todo-keyword-faces
           '(("TODO" . org-warning) ("STARTED" . "yellow")
             ("CANCELED" . (:foreground "blue" :weight bold))))

org-faces-easy-properties
\subsubsection{dependencies}
\label{sec:org477efcc}
org-enforce-todo-dependencies  block done while children undone
property ORDERED blocked until earlier siblings done.

:ORDERED: t

:NOBLOCKING: t

‘C-c C-x o     (org-toggle-ordered-property)’
‘C-u C-u C-u C-c C-t’ circumventing any state blocking

org-depend.el  more complex dependency structures.
\subsection{Progress logging}
\label{sec:org94a6ea5}
\subsubsection{{\bfseries\sffamily DONE} Closing items}
\label{sec:org706325a}
(setq org-log-done 'time)
(setq org-log-done 'note)

‘\#+STARTUP: logdone’
‘\#+STARTUP: lognotedone’.
\subsubsection{Tracking TODO state changes}
\label{sec:org30de272}
newest first.  org-log-states-order-reversed
org-log-into-drawer
LOGBOOK
LOG\(_{\text{INTO}}\)\(_{\text{DRAWER}}\)

‘!’ (for a timestamp) 
‘@’ (for a note with timestamp)

(setq org-todo-keywords
    '((sequence "TODO(t)" "WAIT(w@/!)" "|" "DONE(d!)" "CANCELED(c@)")))
/!
the note taken when entering the state, 
a timestamp should be recorded when leaving the WAIT state.

\begin{itemize}
\item TODO Log each state with only a time
:LOGGING: TODO(!) WAIT(!) DONE(!) CANCELED(!)
\item TODO Only log when switching to WAIT, and when repeating
:LOGGING: WAIT(@) logrepeat
\item TODO No logging at all
:LOGGING: nil
\end{itemize}
\subsubsection{Tracking your habits}
\label{sec:org582f068}
\begin{enumerate}
\item {\bfseries\sffamily TODO} Shave
\label{sec:org0d66577}
‘org-habit-graph-column’
‘org-habit-preceding-days’
‘org-habit-following-days’
‘org-habit-show-habits-only-for-today’
\end{enumerate}
\subsection{{\bfseries\sffamily TODO} Priorities}
\label{sec:org1ec7b87}
TODO [\#A] headline
A B(without a cookie is treated like B) C
org-priority-faces

\begin{itemize}
\item ‘C-c ,’   (‘org-priority’)
\item ‘S-<UP>     (org-priority-up)’
\item ‘S-<DOWN>     (org-priority-down)’
\begin{itemize}
\item org-priority-start-cycle-with-default
\end{itemize}
\end{itemize}

‘org-highest-priority’
‘org-lowest-priority’
‘org-default-priority’
\subsection{Breaking down tasks}
\label{sec:org7c2d306}
[/] [\%] overview
C-c C-c update each time

\begin{itemize}
\item COOKIE\(_{\text{DATA}}\)
\begin{itemize}
\item ‘checkbox’
\item ‘todo’
\end{itemize}
\end{itemize}

‘org-hierarchical-todo-statistics’.  
:COOKIE\(_{\text{DATA}}\): todo recursive

automatically to done when children are done :
(defun org-summary-todo (n-done n-not-done)
       "Switch entry to DONE when all subentries are done, to TODO otherwise."
       (let (org-log-done org-log-states)   ; turn off logging
         (org-todo (if (= n-not-done 0) "DONE" "TODO"))))

(add-hook 'org-after-todo-statistics-hook 'org-summary-todo)
\subsection{Checkboxes}
\label{sec:orgb574004}
\begin{itemize}
\item ‘C-c C-c     (org-toggle-checkbox)’
\begin{itemize}
\item C-u C-u +  [-]
\end{itemize}
\item ‘C-c C-x C-b     (org-toggle-checkbox)’
\begin{itemize}
\item C-u +  checkbox presence
\item C-u C-u  [-]
\item active region:
\item cursor in a headline:
\item no active:
\end{itemize}
\item ‘M-S-<RET>     (org-insert-todo-heading)’
\item ‘C-c C-x o     (org-toggle-ordered-property)’
\begin{itemize}
\item org-track-ordered-property-with-tag
\end{itemize}
\item ‘C-c \#     (org-update-statistics-cookies)’
\begin{itemize}
\item C-u +  update the entire file
\end{itemize}
\end{itemize}
\section{Tags}
\label{sec:orgd83fbbe}
\subsection{\^{}}
\label{sec:org883b147}
\subsection{Tag inheritance}
\label{sec:org61b8022}
org-tags-exclude-from-inheritance
org-use-tag-inheritance
org-tags-match-list-sublevels
org-agenda-use-tag-inheritance
\subsection{Setting tags}
\label{sec:orgcb2de26}
:tag
M-<TAB> completion
\begin{itemize}
\item ‘C-c C-q     (org-set-tags-command)’
\begin{itemize}
\item org-tags-column
\item C-u +  realign
\end{itemize}
\item ‘C-c C-c     (org-set-tags-command)’

\item list of tags:
\begin{itemize}
\item dynamically: used in current buffer.
\item globally: org-tag-alist
\item given file:
\item \#+TAGS: ::use dynamic taglist
\end{itemize}
\end{itemize}

org-tag-persistent-alist
\begin{itemize}
\item fast tag selection:
\begin{itemize}
\item (setq org-tag-alist '(("@work" . ?w) ("@home" . ?h) ("laptop" . ?l)))
\item \#+TAGS: @work(w)  @home(h)  @tennisclub(t)  laptop(l)  pc(p)
\item \#+TAGS: @work(w)  @home(h)  @tennisclub(t) \n laptop(l)  pc(p)
\begin{itemize}
\item \#+TAGS: @work(w)  @home(h)  @tennisclub(t)
\end{itemize}
\item \#+TAGS: \{ @work(w)  @home(h)  @tennisclub(t) \}  laptop(l)  pc(p)
\end{itemize}
\end{itemize}
C-c C-c activate any changes

mutually exclusive groups in the variable ‘org-tag-alist’,
‘:startgroup’ ‘:endgroup’
‘:newline’

org-fast-tag-selection-single-key
\subsection{Tag hierarchy}
\label{sec:org348e898}
(setq org-tag-alist '((:startgrouptag)
                           ("GTD")
                           (:grouptags)
                           ("Control")
                           ("Persp")
                           (:endgrouptag)
                           (:startgrouptag)
                           ("Control")
                           (:grouptags)
                           ("Context")
                           ("Task")
                           (:endgrouptag)))
mutually exclusive:
‘org-tag-alist’:
‘:startgroup’ ‘:endgroup’
‘:startgrouptag’ \& ‘:endgrouptag’

org-toggle-tags-groups
org-group-tags
\subsection{Tag searches}
\label{sec:orgc537d96}
\begin{itemize}
\item ‘C-c / m  or  C-c $\backslash$     (org-match-sparse-tree)’
\begin{itemize}
\item + C-u ignore not TODO line
\end{itemize}
\item ‘C-c a m     (org-tags-view)’
\item ‘C-c a M     (org-tags-view)’
\end{itemize}
\section{Properties and columns}
\label{sec:org2c66813}
\subsection{Property syntax}
\label{sec:orgbff0698}
Properties are key-value pairs.
org-use-property-inheritance

:xyz: one of :xyzALL:

C-c C-c to activete this change.

org-global-properties

‘M-<TAB>     (pcomplete)’
‘C-c C-x p     (org-set-property)’
‘C-u M-x org-insert-drawer <RET>’
‘C-c C-c     (org-property-action)’
‘C-c C-c s     (org-set-property)’
‘S-<RIGHT>     (org-property-next-allowed-value)’
‘S-<LEFT>     (org-property-previous-allowed-value)’
‘C-c C-c d     (org-delete-property)’
‘C-c C-c D     (org-delete-property-globally)’
‘C-c C-c c     (org-compute-property-at-point)’
\subsection{Special properties}
\label{sec:orgb23c3c1}
\subsection{Property searches}
\label{sec:orgf1cb5e3}
‘C-c / m  or  C-c $\backslash$     (org-match-sparse-tree)’
‘C-c a m     (org-tags-view)’
‘C-c a M     (org-tags-view)’
‘C-c / p’
\subsection{Property inheritance}
\label{sec:org84b3d43}
org-use-property-inheritance
hard code:
‘COLUMNS’
‘CATEGORY’
‘ARCHIVE’
‘LOGGING’
\subsection{Column view}
\label{sec:org87f789e}
\subsubsection{Defining columns}
\label{sec:orge6f7124}
\begin{enumerate}
\item Scope of column definitions
\label{sec:org3333ef2}
:COLUMNS: \%25ITEM \%TAGS \%PRIORITY \%TODO
\item Column attributes
\label{sec:org7b8d0d1}
\%[WIDTH]PROPERTY[(TITLE)][\{SUMMARY-TYPE\}]
\end{enumerate}
\subsubsection{Using column view}
\label{sec:org0719116}
\begin{itemize}
\item ‘C-c C-x C-c     (org-columns)’
\begin{itemize}
\item org-columns-default-format
\item ‘r     (org-columns-redo)’
\item ‘g     (org-columns-redo)’
\item ‘q     (org-columns-quit)’
\item ‘<LEFT> <RIGHT> <UP> <DOWN>’ move
\item ‘S-<LEFT>/<RIGHT>’ swithc value
\item ‘1..9,0’  select value
\item ‘n     (org-columns-next-allowed-value)’
\item ‘p     (org-columns-previous-allowed-value)’
\item ‘e     (org-columns-edit-value)’
\item ‘C-c C-c     (org-columns-set-tags-or-toggle)’
\item ‘v     (org-columns-show-value)’
\item ‘a     (org-columns-edit-allowed)’
\item ‘<     (org-columns-narrow)’
\item ‘>     (org-columns-widen)’
\item ‘S-M-<RIGHT>     (org-columns-new)’
\item ‘S-M-<LEFT>     (org-columns-delete)’
\end{itemize}
\end{itemize}
\subsubsection{Capturing column view}
\label{sec:orgfeeef3e}
\begin{itemize}
\item ‘:id’
\begin{itemize}
\item local
\item global
\item "ID"
\begin{itemize}
\item M-x org-id-copy <RET>
\end{itemize}
\end{itemize}
\item ‘:hlines’
\item ‘:vlines’
\item ‘:maxlevel’
\item ‘:maxlevel’
\item ‘:indent’

\item ‘C-c C-x i     (org-insert-columns-dblock)’
\end{itemize}
‘C-c C-c  or  C-c C-x C-u     (org-dblock-update)’
‘C-u C-c C-x C-u     (org-update-all-dblocks)’

org-collector
\subsection{Property API}
\label{sec:orgc9fa218}

\section{Dates and times}
\label{sec:orge6723d6}
\subsection{Timestamps}
\label{sec:org3d32a52}
PLAIN TIMESTAMP; EVENT; APPOINTMENT
\begin{itemize}
\item Meet Peter at the movies
\textit{<2006-11-01 三 19:15>}
\end{itemize}
TIMESTAMP WITH REPEATER INTERVAL
\begin{itemize}
\item Pick up Sam at school
\textit{<2007-05-16 三 12:30 +1w>}
\end{itemize}
DIARY-STYLE SEXP ENTRIES
\begin{itemize}
\item 22:00-23:00 The nerd meeting on every 2nd Thursday of the month
\textit{<\%\%(diary-float t 4 2)>}
\end{itemize}
TIME/DATE RANGE
          ** Meeting in Amsterdam
             \textit{<2004-08-23 一>--<2004-08-26 四>}
INACTIVE TIMESTAMP
\begin{itemize}
\item Gillian comes late for the fifth time
\textit{[2006-11-01 三]}
\end{itemize}
\subsection{Creating timestamps}
\label{sec:orgbd5119a}
\subsubsection{\^{}}
\label{sec:org5b89c34}
\begin{itemize}
\item ‘C-c .     (org-time-stamp)’
\begin{itemize}
\item twice in succession, a time range is inserted.
\end{itemize}
\item ‘C-c !     (org-time-stamp-inactive)’
\item ‘C-u C-c .’ date and time
\item ‘C-u C-c !’ date and time
\item ‘C-c C-c’ Normalize timestamp
\item ‘C-c <     (org-date-from-calendar)’
\item ‘C-c >     (org-goto-calendar)’
\item ‘C-c C-o     (org-open-at-point)’
\item ‘S-<LEFT>     (org-timestamp-down-day)’
\item ‘S-<RIGHT>     (org-timestamp-up-day)’
\item ‘S-<UP>     (org-timestamp-up)’
\item ‘S-<DOWN>     (org-timestamp-down-down)’
\item ‘C-c C-y     (org-evaluate-time-range)’
\begin{itemize}
\item with a prefix argument, insert result
\end{itemize}
\end{itemize}
\subsubsection{The date/time prompt}
\label{sec:org54dd5ee}
3-2-5         ⇒ 2003-02-05
+0            ⇒ today

‘parse-time-months’ and ‘parse-time-weekdays’.
org-read-date-force-compatible-dates

11am-1:15pm    ⇒ 11:00-13:15

You can control the calendar fully from the minibuffer:
     <RET>              Choose date at cursor in calendar.
     mouse-1            Select date by clicking on it.
     S-<RIGHT>/<LEFT>   One day forward/backward.
     S-<DOWN>/<UP>      One week forward/backward.
     M-S-<RIGHT>/<LEFT> One month forward/backward.
     > / <              Scroll calendar forward/backward by one month.
     M-v / C-v          Scroll calendar forward/backward by 3 months.
     M-S-<DOWN>/<UP>    Scroll calendar forward/backward by one year.
\subsubsection{Custom time format(not recomend)}
\label{sec:org51c56b2}
‘org-display-custom-times’
‘org-time-stamp-custom-formats’
‘C-c C-x C-t     (org-toggle-time-stamp-overlays)’
\subsection{Deadlines and scheduling}
\label{sec:org8cffed2}
\subsubsection{\^{}}
\label{sec:org1b20a56}
DEADLINE
org-deadline-warning-days
org-agenda-skip-deadline-prewarning-if-scheduled
SCHEDULED
‘org-scheduled-delay-days’
‘org-agenda-skip-scheduled-delay-if-deadline’
\subsubsection{Inserting deadline/schedule}
\label{sec:org7fe3b87}
\begin{itemize}
\item ‘C-c C-d     (org-deadline)’
\begin{itemize}
\item C-u +  remove
\item org-log-redeadline
\item ‘\#+STARTUP’ keywords ‘logredeadline’,‘lognoteredeadline’, and ‘nologredeadline’
\end{itemize}
\item ‘C-c C-s     (org-schedule)’
\begin{itemize}
\item C-u + remove
\item org-log-reschedule
\item ‘\#+STARTUP’ keywords ‘logreschedule’,‘lognotereschedule’, and ‘nologreschedule’
\end{itemize}
\item ‘C-c / d     (org-check-deadlines)’
\begin{itemize}
\item org-deadline-warning-days
\item C-u + show all in file
\item C-N + show N days
\end{itemize}
\item ‘C-c / b     (org-check-before-date)’
\item ‘C-c / a     (org-check-after-date)’
\end{itemize}
‘org-schedule’
‘org-deadline’
\subsubsection{Repeated tasks}
\label{sec:org3644891}
‘DEADLINE:\textit{<2005-10-01 六 +1m -3d>}’
\sout{+
.}
the variable ‘org-agenda-skip-scheduled-if-deadline-is-shown’ 
‘repeated-after-deadline’
C-c C-x c  copy subtree
\subsection{Clocking work time}
\label{sec:orgb2c5ab8}
\subsubsection{{\bfseries\sffamily DONE} Clocking commands}
\label{sec:org164453a}
\begin{itemize}
\item ‘C-c C-x C-i     (org-clock-in)’
\begin{itemize}
\item org-clock-into-drawer
\begin{itemize}
\item ‘CLOCK\(_{\text{INTO}}\)\(_{\text{DRAWER}}\)’ ‘LOG\(_{\text{INTO}}\)\(_{\text{DRAWER}}\)’
\end{itemize}
\item C-u + recently clocked tasks
\item C-u C-u + mark as default task      letter 'd'
\item C-u C-u C-u  continuous clocking
\item CLOCK\(_{\text{MODELINE}}\)\(_{\text{TOTAL}}\)
\end{itemize}
\item ‘C-c C-x C-o     (org-clock-out)’
\begin{itemize}
\item org-log-note-clock-out
\begin{itemize}
\item ‘\#+STARTUP: lognoteclock-out’
\end{itemize}
\end{itemize}
\item ‘C-c C-x C-x     (org-clock-in-last)’
\begin{itemize}
\item C-u + select
\item C-u C-u continuous clocking
\end{itemize}
\item ‘C-c C-x C-e     (org-clock-modify-effort-estimate)’
\item ‘C-c C-c  or  C-c C-y     (org-evaluate-time-range)’
\item ‘C-S-<up/down>     (org-clock-timestamps-up/down)’
\item ‘S-M-<up/down>     (org-timestamp-up/down)’
\item ‘C-c C-t     (org-todo)’
\item ‘C-c C-x C-q     (org-clock-cancel)’
\item ‘C-c C-x C-j     (org-clock-goto)’
\begin{itemize}
\item ‘C-u’ + select task
\end{itemize}
\item ‘C-c C-x C-d     (org-clock-display)’
\begin{itemize}
\item ‘org-remove-highlights-with-change’)
\item ‘C-c C-c’.
\end{itemize}
\end{itemize}
\subsubsection{The clock table}
\label{sec:org3d67950}
\begin{itemize}
\item ‘C-c C-x C-r     (org-clock-report)’
\begin{itemize}
\item in existing clock table just update it
\item C-u first clock
\end{itemize}
\item ‘C-c C-c  or  C-c C-x C-u     (org-dblock-update)’
\item ‘C-u C-c C-x C-u’
\end{itemize}
‘S-<LEFT>’
‘S-<RIGHT>     (org-clocktable-try-shift)’

org-clocktable-defaults
org-clocktable-write-default
\subsubsection{Resolving idle time}
\label{sec:orge362076}
\begin{enumerate}
\item Resolving idle time
\label{sec:orgf47cbaa}
 org-clock-idle-time
‘M-x org-resolve-clocks <RET>’ (or‘C-c C-x C-z’).
k K s S C
\item Continuous clocking
\label{sec:orge9e4da5}
org-clock-continuously
three uiversal arguments with ‘org-clock-in’ and 
two ‘C-u C-u’ with ‘org-clock-in-last’.
\end{enumerate}
\subsection{Effort estimates}
\label{sec:org35bebb3}
Effort estimates are stored in a special property ‘EFFORT’.
‘C-c C-x e     (org-set-effort)’
‘C-c C-x C-e     (org-clock-modify-effort-estimate)’
\subsection{Timers}
\label{sec:org46f5868}
Relative timer and countdown timer.
\begin{itemize}
\item ‘C-c C-x 0     (org-timer-start)’
\item ‘C-c C-x ;     (org-timer-set-timer)’
\item ‘C-c C-x .     (org-timer)’
\begin{itemize}
\item C-u + relative timer restarted
\end{itemize}
\item ‘C-c C-x -     (org-timer-item)’
\begin{itemize}
\item C-u first reset the relative timer to 0.
\end{itemize}
\item ‘M-<RET>     (org-insert-heading)’
\item ‘C-c C-x ,     (org-timer-pause-or-continue)’
\item ‘C-c C-x \_     (org-timer-stop)’
\end{itemize}
\section{Capture - Refile - Archive}
\label{sec:orgaa48992}
\subsection{Capture}
\label{sec:org78cb416}
\subsubsection{\^{}}
\label{sec:org7692ac2}
org-capture.el org-capture-templates
\subsubsection{Setting up capture}
\label{sec:org84d9b94}
(setq org-default-notes-file (concat org-directory "/notes.org"))
(define-key global-map "\C-cc" 'org-capture)
\subsubsection{Using capture}
\label{sec:org5a7057f}
\begin{itemize}
\item ‘C-c c     (org-capture)’ need to installl it.
\begin{itemize}
\item ‘C-u C-c c’ visit the capture
\item ‘C-u C-u C-c c’ visit last capture
\item C-0 + insert at current buffer.
\end{itemize}
\item ‘C-c C-c     (org-capture-finalize)’
\begin{itemize}
\item C-u jump
\end{itemize}
\item ‘C-c C-w     (org-capture-refile)’
\begin{itemize}
\item C-u passed on to the 'org-refile' command.
\end{itemize}
\item ‘C-c C-k     (org-capture-kill)’
\end{itemize}
org-capture-last-stored
\subsubsection{Capture templates}
\label{sec:org155f4c7}
‘C-c c C’  ‘org-capture-templates’.
\begin{enumerate}
\item Template elements
\label{sec:orgf25d2d2}
\item Template expansion
\label{sec:org39eb0fb}
\item Templates in contexts
\label{sec:org6686b9c}
\end{enumerate}
\subsection{Attachments\hfill{}\textsc{ATTACH}}
\label{sec:org4d15241}
‘C-c C-a     (org-attach)’
\subsection{RSS feeds}
\label{sec:org8f6083c}
\subsection{Protocols}
\label{sec:org97f16b0}
\subsection{Refile and copy}
\label{sec:org6a94f22}
\subsection{Archiving}
\label{sec:orga3023a3}
\section{Agenda views}
\label{sec:org61b061d}
\section{Markup}
\label{sec:org1bf9702}
\subsection{Embedded \LaTeX{}}
\label{sec:org27d5ef9}
\LaTeX{} is widely used to typeset scientific documents.
\subsubsection{\LaTeX{} fragments}
\label{sec:org2de6f79}
When exporting to HTML, Org can use either MathJax or transcode the math
into images.

sudo apt install transcode:
\begin{itemize}
\item Environments of any kind. \bengin appears in newline.
\item Text within the usual \LaTeX{} math delimiters.
\begin{itemize}
\item \(no space\)
\item use \(...\) when in doubt
\end{itemize}
\end{itemize}

example:

\begin{equation}
x=\sqrt{b}
\end{equation}

\(a^2=b\) 
\(b=2\)

$$ a=+\sqrt{2} $$
\[ a=-\sqrt{2} \].

org-export-with-latex

Do the right thing automatically (MathJax)

\subsubsection{Previewing \LaTeX{} fragments}
\label{sec:org90d8670}
If you have a working \LaTeX{} installation and ‘dvipng’, ‘dvisvgm’ or
‘convert’ installed.
sudo apt install dvipng


org-format-latex-options
org-format-latex-header

:scale
:html-scale

‘C-c C-x C-l’
‘C-c C-c’
\subsubsection{CDLaTeX mode}
\label{sec:orgc703a2c}
CDLaTeX mode is a minor mode that is normally used in combination with a
major \LaTeX{} mode like AUCTeX in order to speed-up insertion of environments
and math templates.
\section{Exporting}
\label{sec:org4929232}
\section{Publishing}
\label{sec:orgf4ad730}
\section{Working with source code}
\label{sec:orgc0a3521}
\url{workingWithSourceCode.org}
\section{Miscellaneous}
\label{sec:org9c7a1d4}
\section{Hacking}
\label{sec:org0bc7696}
\section{MobileOrg}
\label{sec:org1725b44}
\section{History and acknowledgments}
\label{sec:org6e89321}
\section{GNU Free Documentation License}
\label{sec:org9f20bef}
\section{Main Index}
\label{sec:orgb7a6f92}
\section{Key Index}
\label{sec:orgabd180d}
\section{Command and Function Index}
\label{sec:org10878f9}
\section{Variable Index}
\label{sec:orgc9b516a}
\end{document}
